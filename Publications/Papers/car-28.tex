\documentclass{article}
\usepackage{a4wide,german,url}
\newcommand{\SD}{{\sc Symbolic\-Data}}

\begin{document}

\section*{\centering Neues vom \SD-Projekt}

\begin{center} 
Hans-Gert Gr"abe (Leipzig)\footnote{published in
``Computeralgebra-Rundbrief'' 28, March 2001}
\end{center}

Im CAR 26 (M"arz 2000) wurde das \SD-Projekt ausf"uhrlich vorgestellt,
wobei der Schwerpunkt auf der Darstellung der Konzepte lag, nach denen
die Datensammlung sowie die zugeh"origen Werkzeuge organisiert sind.

Inzwischen haben wir die vorgestellten Konzepte weiter praktisch
umgesetzt und insbesondere die Dokumentation grundlegend verbessert.
Dazu entwarf und implementierte O.~Bachmann (Kaiserslautern) ein
Konzept dynamischer Links, das es gestattet, "uber eine erweiterte
HTML-Syntax neue Webseiten leicht in das bestehende
Dokumentationsgef"uge einzubauen.  O.~Bachmann hat mit Wechsel seiner
Arbeitsstelle das \SD-Projekt zum Ende 2000 verlassen.

Von mir wurde die bereits im Konzept angelegte Trennung von Daten- und
Programmteil (vgl. CAR 26) nun auch auf technischer Ebene realisiert,
wodurch es m"oglich ist, dieselben Werkzeuge f"ur verschiedene Projekte
zur Verwaltung symbolischer Informationen einzusetzen. Die jeweils
verwendete Datenbasis kann dazu "uber den Wert einer Option eingestellt
werden.  

Zur Verwaltung symbolischer Daten kann zwar auch eine herk"ommliche
Datenbank eingesetzt werden, jedoch erlaubt unser objekt-relationaler
Ansatz zusammen mit der Flexibilit"at von Perl einen deutlich
einfacheren Umgang mit ASCII-Quellen (bis hin zur M"oglichkeit, diese
'in situ' zu korrigieren), so dass sich die \SD-Werkzeuge auch
``artfremd'' (ich verwende sie zum Aufbau einer Literatur-Datenbank
sowie beim Management von Olympiade-Aufgaben) einsetzen lassen.
\medskip

Die Resonanz auf unsere Arbeit innerhalb der CA-Gemeinde ist jedoch
immer noch gering.  Ich weise deshalb noch einmal auf die Anliegen
hin, die wir mit unserem Projekt verfolgen: 
\begin{enumerate}
\item[1.] Wir wollen die Bem"uhungen verschiedener Gruppen zur
Erstellung von Perl-Werkzeugen zum Management digitaler symbolischer
Daten aus verschiedenen Bereichen der Computeralgebra vereinigen.
\end{enumerate}

Wir gehen davon aus, dass auch andere Gruppen sich eigene Werkzeuge
f"ur Vergleichs- oder Testrechnungen erstellt haben oder erstellen und
hier bereits mehrfach das Fahrrad neu erfunden wurde und auch weiter
erfunden werden wird, so dass es an der Zeit ist, das ganze know how
einmal zu sichten und zu b"undeln.

Unsere Werkzeuge (provided 'as is') k"onnen von der \SD-Webseite
\url{http://www.symbolicdata.org} herunter geladen und leicht f"ur
spezielle Zwecke in einem lokalen Projekt angepasst und modifiziert
werden.  Eine sinnvolle Weiterentwicklung ist erst auf der Basis eines
so gewonnenen Erfahrungsschatzes m"oglich.

\begin{enumerate}
\item[2.] \SD\ stellt ein zentrales Repositorium zur Verf"ugung, in dem
digitale Benchmark-Daten aus verschiedenen Bereichen der
Computeralgebra gesammelt werden (k"onnen).
\end{enumerate}

Ein solcher {\em upload} ist derzeit nur "uber eine direkte Beteiligung
am Projekt m"oglich, in dessen Rahmen Zugang zu unserem CVS-Repository
am UMS MEDICIS in Paris
(\url{http://www.medicis.polytechnique.fr/medicis}) besteht.
\medskip

Eine stabile Version 0.4 der zur Zeit verf"ugbaren Werkzeuge und Daten
ist ab 1.~M"arz~2001 auf unserem Server verf"ugbar.  Teilbereiche des
Projekts warten auf die weitere Ausgestaltung (z.B. ein
Frontend/Backend-System, um die Ladezeiten zu verringern; ein auf
"Ahnlichkeit basierendes System von Vergleichen von Records; dynamische
Webseiten-Generierung aus der Datenbasis; ein vern"unftiger
Select-Mechanismus), wozu weitere Mitstreiter willkommen sind.

\end{document}
