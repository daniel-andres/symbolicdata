\documentclass{llncs}
\usepackage{url}

\newenvironment{code}{\tt \begin{tabbing}
\hskip12pt\=\hskip12pt\=\hskip12pt\=\hskip12pt\=\hskip5cm\=\hskip5cm\=\kill}
{\end{tabbing}}

\newcommand{\SD}{{\sc Symbolic\-Data}}
\def\pw{{\char94}}
\def\dq{{\char34}}

\title{The {\SD} Project -- Towards a Computer Algebra Social Network}
\author{Hans-Gert Gr\"abe \and Andreas Nareike\and Simon Johanning}
\institute{Universit\"at Leipzig, Germany\\
\email{(graebe|nareike|johanning)@informatik.uni-leipzig.de}}
\begin{document}
\maketitle

\begin{abstract}
  We report about a complete redesign of the tools and data of the {\SD}
  project according to RDF technologies and Linked Data principles that proved
  to be powerful within modern semantic web approaches. During that redesign
  the focus of the project changed from a mere data store towards the vision
  of a Computer Algebra Social Network (CASN) to support technically
  intercommunity communication between Computer Algebra subcommunities.  In
  the last part of the paper we describe ongoing efforts to implement a
  technical basis for a Distributed Semantic Social Network infrastructure to
  run such a CASN.
\end{abstract}

\section{Introduction}

The SymbolicData project grew up from the Special Session on Benchmarking at
the 1998 ISSAC conference to continue the efforts started by the PoSSo
\cite{PoSSo} an FRISCO \cite{FRISCO} projects. It aimed at building a reliable
and sustainably available collection of Polynomial Systems that were reported
in the literature for benchmarking and profiling of CA software, to extend and
update it, to collect meta information about the records, and also to develop
tools to manage the data and to set up and run reliable tests and benchmark
computations on the data.  A first prototype was developed during 1999--2002
by Olaf Bachmann and Hans-Gert Gr\"abe with data from \emph{Polynomial Systems
  Solving} and \emph{Geometry Theorem Proving}.

There was almost no advance during 2002--2005. In a second phase around 2006
the project matured again and extended its scope. Data was supplied by the
CoCoA group (F.~Cioffi), the Singular group (M.~Dengel, M.~Brickenstein,
S.~Steidel, M.~Wenk), V.~Levandovskyy (non commutative polynomial systems,
G-Algebras) and R.~Hemmecke (Test sets from Integer Programming). In 2005 the
German Fachgruppe Computeralgebra launched the Web site
\url{http://www.symbolicdata.org}.  During the Special Semester on Gr\"obner
Bases (GB) in March 2006 we tried to join forces with the GB-Bibliography
project (B. Buchberger, A. Zapletal) and the GB-Facilities project
(V. Levandovskyy).

In 2009 we started to refactor the data along standard Semantic Web concepts
based on the Resource Description Framework (RDF).  We completed a redesign of
the data along RDF based semantic technologies, set up a Virtuoso
\cite{Virtuoso} based RDF triple store and SPARQL endpoint at
\url{http://www.symbolicdata.org} as an open data service along Linked Data
standards \cite{lod}, and started both conceptual and practical work towards a
semantic-aware Computer Algebra Social Network. The new {\SD} data and tools
were released as version~3 in September 2013.

This and ongoing work was realised within a project \emph{Benchmarking in
  Symbolic Computations and Web 3.0} supported by the \emph{Saxonian E-Science
  Initiative} \cite{E-Science-Sachsen} with a 12 months grant for Andreas
Nareike in 2012/13 and a five months grant for Simon Johanning in 2014.

One of the main decisions within that redesign process was a non-technical one
-- leave the focus on data storage and the ``roots'' within the Polynomial
Systems Solving CA subcommunity in favour of stronger social interlinking. We
reshaped the {\SD} Project as \emph{intercommunity project}, that addresses
needs of subcommunities within the Symbolic Computation community to profile,
test and benchmark implementations as a \emph{cross cutting
  activity}\footnote{See
  \url{http://en.wikipedia.org/wiki/Cross-cutting_concern}.}, and started to
develop links to other intercommunity activities as \emph{sagemath}
\cite{Sagemath}, \emph{lmonade} \cite{lmonade} or \emph{swmath} \cite{swmath}.

In section 2 and 3 we describe the current {\SD} infrastructure in more
detail.  The rest of this paper addresses conceptual and practical problems,
experiences and solutions towards a semantic-aware Computer Algebra Social
Network as an intercommunity project.

\section{The {\SD} Infrastructure}

Our basic resources (examples for testing, profiling and benchmarking software
and algorithms from different areas of symbolic computation) are publicly
available in XML markup, meta data in RDF notation both from a public git
repo, hosted at \texttt{http://github.org/symbolicdata}, and from our remote
RDF triple store at \url{http://symbolicdata.org/Data}. Moreover, we offer a
SPARQL endpoint \cite{sdsparql} to explore the data by standard Linked Data
methods.

The website operates on a standardized installation using an Apache web server
to deliver the data, the Virtuoso RDF data store \cite{Virtuoso} as data
backend, a SPARQL endpoint and (optionally) OntoWiki \cite{OntoWiki} to
explore, display and edit the data. This installation can easily be rolled out
on a local site\footnote{Tested with Linux Debian and Ubuntu 12.04 LTS
  standard distributions; a more detailed description can be found in the
  {\SD} wiki \cite{sdwiki}.} to support local testing, profiling and
benchmarking.

The distribution contains also tools and prototypical solutions for a local
compute environment as, e.g., provided by Sagemath \cite{Sagemath}.  The
Python based \emph{SDEval package} \cite{sdeval} by Albert Heinle offers a
JUnit like framework to set up, run, log, monitor and interrupt testing and
benchmarking computations. The \emph{SDSage package} \cite{sdsage} by Andreas
Nareike provides a showcase for {\SD} integration with the Sagemath compute
environment.

We follow a development process along the Integration-Manager-Workflow
Model\footnote{See \url{http://git-scm.com/book/en/Distributed-Git-Distributed-Workflows}.}.
This makes it easy to join forces with the {\SD} team: Fork the repo to your
github account, start development and send a pull request to the Integration
Manager if you think you produced something worth to be integrated into the
upstream master branch.  Even if your contribution is not pulled to the
upstream, people can use it, since they can pull it from your github repo to
their github repo. This allows even for agile common small feature development
-- a widely practised way to advance projects hosted at \texttt{github.com}.
You are encouraged to start a discussion about your plans early in the process
and regularly report your progress on the {\SD} mailing list.

Currently the {\SD} data collection contains resources from Polynomial Systems
Solving (390 records, 633 configurations), Free Algebras (83 records),
G-Algebras (8 records), GeoProofSchemes (297 records) and Test Sets from
Integer Programming (28 records). These resources are stored in a flat XSchema
based XML syntax developed within {\SD} version 2 that uses well established
intracommunity syntaxes for the internal data.

\section{Towards a Decentralized Infrastructure of Resources}

Note that RDF provides a strong conceptual distinction between
\emph{resources} (basic information) and \emph{resource descriptions} (meta
information) and with {\SD} version 3 we use XML representations more
concisely to focus on the basic information structure itself.

Since the basic information is provided by different CA subcommunities it is a
good advice to use the (textual) syntactical notations well established within
a subcommunity to store data. In most cases such syntactical notations are not
XML based, so one cannot use a standard XML parser to parse the internal
textual representations ``out of the box''.

Since the subcommunity has plenty of parsers and tools at hand to input
textual representations into their \emph{semantic-aware tools} this is not a
real obstacle in practise. In particular, in the early times of {\SD} we had a
dispute about representation of polynomials -- use the well established
operator syntax as, e.g., \texttt{x{\pw}3+5*x-2}, or provide polynomials in
XML-based OpenMath or MathML syntax. We decided to store polynomials in the
compact human readable operator syntax, as in the PoSSo project.

In the {\SD} data structure concept we use XML markup mainly to compile
heterogeneously structured data into a single resource in such a way that the
different parts of this data can be extracted by a standard XML parser and
passed to appropriate semantic-aware tools for further processing.  To start a
new data collection within {\SD} you have to decide about the parts of data
that have to be bundled for a single resource, to decide about the syntactical
representation of these parts according to the standards of your scientific
subcommunity, to develop a XSchema based XML representation for the bundling,
and provide data along that standard.  The main point about the resources is
the reliable and sustainable availability of the data through a
\emph{permanent web address}, a \emph{Unique Resource Identifier} (URI).  We
provide access to our centrally managed resources via
\url{http://symbolicdata.org/XMLResources/}.

Such a concept is not restricted to centrally managed resources, but can
easily be extended to other data stores on the web that are operated by
diffe\-rent CA subcommunities and offer a minimum of Linked Data facilities.
There are draft versions of resource descriptions about Fano Polytopes (8630
records) and Birkhoff Polytopes (5399 records) hosted by Andreas Paffenholz
and about Transitive Groups (3605 records) from the Database for Number Fields
of J\"urgen Kl\"uners and Gunter Malle that point to such external resources.  

\section{Resources and Resource Descriptions}

Preparing {\SD} version 3 we decided to strengthen the part of intercommunity
communication aspects.  From this point of view \emph{resources} are owned and
maintained by different CA subcommunities, and meta data or \emph{resource
  descriptions} are important for technically supported interchange of data
between such subcommunities and for intercommunity communication, and hence
should be managed and maintained within a cooperative intercommunity process.

A first question to be solved was about data representation for resources and
resource descriptions.  XML based design principles mainly distinguish between
information (XML records) and information structure (described with XSchema)
and are well suited for data representation of (basic) resources but proved to
be not expressive enough to represent interrelations between different
resources in a flexible way.

\subsection{Why RDF?}

We decided to switch to RDF as basic representation for resource descriptions
by several reasons. First, we could join forces with the Agile Knowledge
Engineering and Semantic Web (AKSW) Group at Leipzig University\footnote{See
  \url{http://aksw.org/About.html}.}, a leading research group in semantic
technologies, and exploit their experience about concepts and tools in that
area. Second, RDF gets established more and more for exchange of meta
information not only within the Linked Open Data world \cite{lod}, but also
for the big projects on standardization of scientific communication as the
Dublin Core DCMI Metadata Terms Initiative \cite{dcterms} or the Joint
Steering Committee for Development of Resource Description and Access
\cite{RDA}.  Third, there are well elaborated concepts and tools how to
exchange RDF based information by a protocol as simple and widely spread as
\emph{HTTP Get} and how to manage that within a standard web server
infrastructure.

\subsection{RDF Basics}

RDF -- the Resource Description Framework -- is about \emph{description} of
\emph{resources}, represented by (globally unique) \emph{resource identifiers}
(URIs).  RDF provides a unified scheme to represent relational information as
\emph{triples}.  There are several notational standards (ntriples, turtle,
rdf/xml, json) for triples and plenty of tools to manage sets of triples,
i.e., \emph{RDF graphs}.

Each such triple can be considered as a \emph{sentence} of a story that
consists of a subject $s$, a predicate $p$ and an object $o$, but different to
real stories the semantics of an RDF graph is that of a \emph{set}, i.e., the
order of the sentences does not matter.  Hence the expressiveness of RDF
stories is very restricted compared to natural languages. The main advantage
however, is a separation between data and search algorithms on data patterns
as in rule based programming.  RDF comes with the standardized pattern based
query language SPARQL to operate such search queries on RDF data stores.  For
{\SD} we use Virtuoso \cite{Virtuoso} as RDF data store and SPARQL endpoint.

RDF has another advantage compared to classical database approaches -- one can
express \emph{descriptions of descriptions}, i.e., database design, within the
same language concepts, and thus share not only descriptions of data but also
descriptions of data descriptions, i.e., information about the semantics of
the data in a machine readable way.

Subjects and predicates have to be URIs while objects (or `values') can be
either URIs or (plain or typed) literals in lexical form (a string included in
quotes). There are some predefined common types (e.g., \texttt{xsd:integer})
but custom types can be defined as well.

A set of triples can be interpreted as a directed graph (RDF graph) with
subjects and objects as nodes (replacing literals by labelled blank nodes) and
predicates as labelled edges between nodes. On the opposite, a directed graph
can be written as a set of triples (and is commonly represented in such a way
as internal data structure of graph programs). Another representation uses
sets of key-value pairs $p \to o$ assigned to the different subjects $s$.
Note that, different to database columns, a key $p$ can have multiple values.

RDF uses some more basic concepts -- \emph{character sets} to compose URIs and
literals and \emph{name spaces} to structure information spaces and to resolve
conflicts within URI creation.  There are more elaborated concepts as OWL,
RDFS etc.\ on top of RDF as explained in the \emph{Semantic Web Stack}
\cite{SWS}, but not yet used within {\SD}. Note that nowadays the syntax layer
below the RDF data interchange layer in the Semantic Web Stack is no more
bound solely to XML as \cite{SWS} might suggest -- the most widespread syntax
representation is in Turtle format\footnote{See
  \url{http://en.wikipedia.org/wiki/Turtle_(syntax)}.}.

\subsection{Linked Data Principles}

The real power of RDF does not originate in an alleged superiority of concepts
but in the \emph{practical availability} of data stores all over the world
that are organized on RDF based principles. Whereas the (traditional) Web 2.0
is build upon interlinked dynamical HTML web pages based on private databases,
the Semantic Web as part of Web 3.0 \cite{SemanticWeb} focuses on interlinking
these private databases themselves into a single big distributed data store.

RDF supports both ways of data dissemination -- by file transfer as in Web 2.0
and by remote access to RDF triple stores as in Web 3.0 --, and so does {\SD}:
You can download whole RDF graphs as data files from our remote host, upload
this data into your local data store and process it locally, but you can also
directly access our remote RDF data store. RDF data stores operate on the HTTP
protocol and hence are best deployed within a webserver infrastructure, either
remote or local. The only difference between remote and local approaches are
the stronger web security requirements for a remote location.

To achieve web access of RDF data on a remote host, URIs should be available
as URLs, i.e., a \emph{HTTP Get} request to an URI should deliver a valuable
portion of RDF information about that subject. This is the core of the Linked
Data Principle \cite{lod} and realised for the
\url{http://symbolicdata.org/Data/} name space within the {\SD} project.

\section{{\SD} Resource Descriptions}

RDF resource descriptions are the main part of the meta information collected
within the {\SD} project. We offer resource descriptions for several purposes
-- resource fingerprints to navigate within the examples, relational
information to, e.g., bibliographical references or CA software descriptions,
and information about activities of people involved with CA research. 

\subsection{Resource Fingerprints}

Semantically equivalent data usually can be given in different syntactical
form. For example, the same Polynomial System can be given with different
variable names, in different polynomial orderings and even in different forms
as, e.g., expanded or factorized polynomials.

To navigate within such data, to prestructure data for efficient search or to
identify a given example within the database it is helpful to precompile
\emph{fingerprints}, i.e., (semantically sound) invariants of the different
examples.  For example, the set of degree lists (in standard grading) or the
set of the lengths of polynomials in distributive normal form are such
invariants for Polynomial Systems. 

Examples with different fingerprints are surely different, examples with the
same fingerprint require more elaborated methods to be distinguished.  In most
cases the latter is not worth to be automated since the ``general nonsense''
knowledge of the experts (optionally added as \texttt{rdfs:comment} to the
resource description) is a more powerful ``tool'' to resolve such
disambiguities.

The computation of fingerprints requires semantic-aware tools and both the
definition of useful fingerprints and its computation are due to the CA
subcommunity experts with the appropriate semantic knowledge and tools.  To
compare user given examples with existing ones it is a good advice to have
enough invariants as fingerprints at hand that can be computed in polynomial
time. 

\subsection{Relational Information}

It was one of the great visions of the {\SD} Project to collect not only
benchmark and testing data but also valuable background information about the
records in the database as, e.g., information about papers, people, history,
systems etc.\ concerned with the examples in our collection.  It was the main
target of {\SD} version 3 to redesign these data along RDF principles.

We provide a general concept of an RDF class \texttt{sd:Annotation} to store
background information in a unified way.  Instances of this class have
predicates
\begin{itemize}
\item \texttt{rdfs:label} -- a label,
\item \texttt{rdfs:comment} -- a text field for annotation,
\item \texttt{sd:relatesTo} -- a set of related URIs.
\end{itemize}

We use that concept in particular to relate \emph{bibliographical information}
of type \texttt{sd:Reference} to different data records.  The management of
bibliographical re\-fe\-ren\-ces was completely redesigned with {\SD}
version~3 exploiting RDF and the established Dublin Core ontology
\cite{dcterms} to represent bibliographical information in a way that is
queryable by standard means and tools. On the other hand, we strongly reduced
the part of information about bibliographical refe\-rences kept inside {\SD}
since there are comprehensive bibliographical stores available on the web that
provide all required information via permanent URIs, although in most cases
not yet in RDF format.  At the moment we provide links to three such
bibliographical stores,
\begin{itemize}
\item the database of Zentralblatt Mathematik (predicate
  \texttt{sd:hasZBentry}), 
\item the Gr\"obner Bases Bibliography database (predicate
  \texttt{sd:hasGBBentry}) and 
\item the citeseer database (predicate \texttt{sd:hasCSentry}).
\end{itemize}

The same applies to information about and references to CA software that is
\SD-internally stored as resource description of type \texttt{sd:CAS} but
points as far as possible to the relevant information within the \emph{swmath}
database \cite{swmath}, even if \emph{swmath} does not (yet) operate by Linked
Open Data standards.

\subsection{Publicly Tracking Personal Profiles}

Bibliographical references, references to CA software and even references
about contributions to {\SD} itself refer to people involved with CA research.
It is one of the challenges of big data stores about scientific publications
to find out all publications of a given author, since the same author may be
listed in different ways in the author list of different publications.  The
big identification projects use elaborated evaluation algorithms of cross
references to solve this problem or -- as the \emph{Zentralblatt Mathematik}
did for a long time -- use simple string pattern matching.  Some time ago the
\emph{Zentralblatt} started a certain kind of tracking of personal
profiles\footnote{See, e.g., the entry
  \url{https://zbmath.org/authors/?q=ai:grabe.hans-gert} of the first author
  of this paper.} to improve that alignment.

We argue that it is a good advise for scientific communities to support such
tracking activities since the benefits much exceed the drawbacks.  Moreover,
active involvement of scientific communities allows to ``track the trackers'',
i.e., to start open discussions and to influence actively the settings of the
tracking process to maximize its benefits and minimize its drawbacks.

The \texttt{sd:Person} database (274 records) supports that process of
disambiguation on the level of references and authors evaluating different
sources of information about CA publications and relating authorship to
\texttt{sd:Person} URIs that are composed following well defined naming rules.
This part of the project is under heavy development with focus on activities
within the German Fachgruppe.

\section{Towards a Computer Algebra Social Network}

From the five stars to be assigned to a Linked Data project according to Tim
Berners-Lee's classification \cite{5stars} {\SD} earned four stars so far (for
offering data in interoperable RDF format on the web and providing a SPARQL
querable RDF triple store).  For the fifth star one has to build up stable
semantic relations to foreign knowledge bases and thus become part of the
Linked Open Data Cloud \cite{lod}.

Much of such interrelation, e.g., a list of interoperability references for
people, software and bibliographical data with \emph{Zentralblatt}, is on the
way.  Moreover, we joined forces with the efforts of the board of the German
Fachgruppe to store and provide information about people and groups working on
CA topics at their new Wordpress driven web site \cite{cafg}.  We developed a
first prototype to store this information in RDF format, to extract it by
means of SPARQL queries and to view it on the web site using the Wordpress
shortcode mechanism\footnote{See \url{http://codex.wordpress.org/Shortcode}.}
via a special Wordpress plugin.  We apply the same technique to maintain
information about upcoming conferences at this site.

The vision of a Computer Algebra Social Network (CASN) goes far beyond that:
Get people involved themselves on a regular basis, set up and run within the
CA community a semantic-aware Facebook like Social Network and contribute to
it about all topics around Computer Algebra using tools that express your
contributions in an RDF based vocabulary that the community agreed upon. This
sounds quite visionary but is in no way utopic. We operate a first
prototypical node of a tool that realizes the challenging concept of a
\emph{Distributed Semantic Social Network} (DSSN) \cite{dssn}.

We set up a second RDF data store at \url{http://symbolicdata.org/casn/} with
information about 
\begin{itemize}
\item upcoming conferences (about 20 entries of type \texttt{sd:Event}) and
\item publications within the ``CA Rundbrief'' of the German Fachgruppe
\end{itemize}
and plan to extend that data store with information about
\begin{itemize}
\item CA projects (initial: projects from the German SPP 1489) and
\item CA working groups (initial: as listed by the German Fachgruppe). 
\end{itemize}
As the project matures this will be interrelated with the DSSN node at
\url{http://symbolicdata.org/xodx/} running a software under development by
the AKSW group in such a way that you can join the CASN and supply your
contributions as you can do (also not yet semantically) in any other social
network.  See our wiki for more information.

\raggedright
\begin{thebibliography}{1}

\bibitem{5stars} Berners-Lee, T.: 5 stars for Open Data.
  \url{http://5stardata.info/} [2014-03-05]

\bibitem{dcterms} DCMI Metadata Terms.
  \url{http://dublincore.org/documents/dcmi-terms/} [2014-02-27]

\bibitem{E-Science-Sachsen} Das eScience-Forschungsnetzwerk Sachsen.
  \newblock \url{http://www.escience-sachsen.de} [2014-02-19]

\bibitem{cafg} Website of the German Fachgruppe Computeralgebra.  
  \url{http://www.fachgruppe-computeralgebra.de/} [2014-03-06]

\bibitem{FRISCO} FRISCO -- A Framework for Integrated Symbolic/Numeric
  Computation. \url{http://www.nag.co.uk/projects/FRISCO.html} [2014-02-19]

\bibitem{sdeval} Heinle, A.: The SDEval framework.  
  \url{http://symbolicdata.org/wiki/SDEval} [2014-02-28]

\bibitem{lod} Linked Data.  \url{http://www.w3.org/DesignIssues/LinkedData.html}
  [2014-05-30]

\bibitem{lmonade} lmonade -- a platform for development and distribution of
  scientific software.  \url{http://www.lmona.de/}
  [2014-05-24]

\bibitem{sdsage} Nareike, A.: The SDSage package.  
  \url{http://symbolicdata.org/wiki/PolynomialSystems.Sage} [2014-02-28]

\bibitem{OntoWiki} Auer, S., Dietzold, S., Lehmann, J., Riechert, T.:
  OntoWiki: A Tool for Social, Semantic Collaboration.  Proceedings of the
  Workshop on Social and Collaborative Construction of Structured Knowledge
  CKC (2007)

  See \url{http://ceur-ws.org/Vol-273/paper_91.pdf} [2014-05-30]

  See also \url{http://aksw.org/Projects/OntoWiki.html} [2014-02-19]

\bibitem{PoSSo} The PoSSo Project. \url{http://posso.dm.unipi.it/}
  [2014-02-19]

\bibitem{RDA} Joint Steering Committee for Development of Resource Description
  and Access. See \url{http://www.rda-jsc.org/rda-new-org.html} or
  \url{http://www.dnb.de/DE/Standardisierung/International/rdaFaq.html}.

\bibitem{Sagemath} Sage -- a free open-source mathematics software system.
  \url{http://www.sagemath.org} [2014-02-19]

\bibitem{SemanticWeb} The Semantic Web.
  \url{http://en.wikipedia.org/wiki/Semantic_Web} [2014-05-24]

\bibitem{SWS} The Semantic Web Stack.
  \url{http://en.wikipedia.org/wiki/Semantic_Web_Stack} [2014-05-24]

\bibitem{swmath} swMATH -- an information service for mathematical software.
  \url{http://www.swmath.org} [2014-05-24]

\bibitem{sdwiki} The {\SD} project wiki.  \url{http://symbolicdata.org/wiki}

\bibitem{sdsparql} The {\SD} SPARQL endpoint.
  \url{http://symbolicdata.org:8890/sparql}

\bibitem{dssn} Tramp, S.\ et al.: DSSN: towards a global Distributed Semantic
  Social Network.  \url{http://aksw.org/Projects/DSSN.html} [2014-03-06]

\bibitem{Virtuoso} Virtuoso Open-Source Edition. \newblock
  \url{http://virtuoso.openlinksw.com/} [2014-02-19]

\end{thebibliography}
\end{document}
