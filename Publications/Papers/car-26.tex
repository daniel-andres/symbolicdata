% contribution to the Computeralgebra-Rundbrief
% $Id: car-26.tex,v 1.2 2002/06/17 09:33:02 graebe Exp $
\documentclass{article}
\usepackage{a4wide,german,url}
\newcommand{\SD}{{\sc Symbolic\-Data}}

\begin{document}

\section*{\centering Das \SD-Projekt} 

\begin{center}
Olaf Bachmann (Kaiserslautern) und Hans-Gert Gr\"abe (Leipzig)
\footnote{published in ``Computeralgebra-Rundbrief'' 26, March 2000}
\end{center}

Wir wollen mit diesem Bericht den Lesern des
Computeralgebra-Rundbriefs das \SD-Projekt, an dem wir im Rahmen der
Benchmarking-Aktivit"aten der Fachgruppe seit etwa einem Jahr
arbeiten, kurz vorstellen.  Wir konzentrieren uns dabei auf die
Darstellung der Motivation, die zu diesem Projekt gef"uhrt hat, und
einige grunds"atzliche "Uberlegungen, aus denen der gew"ahlte Ansatz
deutlich wird.  F"ur konkretere Fragen nach der Struktur der
bisherigen Datensammlung und der erstellten Werkzeuge, der Art, wie
man daraus Benchmark-Rechnungen erzeugen und starten sowie die
Ergebnisse solcher Rechnungen auswerten kann, verweisen wir
interessierte Leser auf die online-Repr"asentation des Projekts auf
unserer Web-Seite \url{www.SymbolicData.org}.  Wie bei einem
Open-Source-Projekt "ublich sind die Daten und Werkzeuge offen
verf"ugbar und k"onnen problemlos heruntergeladen, installiert und
ausprobiert werden.  Wir hoffen, auf diesem Wege neue Mitstreiter zu
finden, die sich der Freiheiten, aber auch der Verantwortung bewusst
sind, die die Mitarbeit an einem solchen Vorhaben mit sich bringt. 

\subsection*{Motivation}

Nach den Diskussionen auf der ISSAC'98 in Rostock "uber
Benchmarking-Aktivit"aten im Bereich der Computeralgebra haben wir
zun"achst begonnen, die bereits existierenden Materialien zu sichten
und zu ordnen. Dabei stellte sich schnell heraus, dass es sinnvoll
ist, diese Materialien in ein einheitliches elektronisches Format zu
"uberf"uhren und geeignete Werkzeuge zur Manipulation dieser
Materialen bereitzustellen.

Entsprechend konzentrierten sich unsere bisherigen Aktivit"aten auf
die Fragestellungen, die im Untertitel des Projekts ``An Electronic
Repository of Tools and Data for Computer Algebra Benchmarks''
deutlich werden.

Die bisherigen Hauptrichtungen des \SD-Projekts lassen sich wie folgt
umrei"sen:
\begin{enumerate}
\item Systematische Sammlung existierender Benchmark-Daten zu
verschiedenen Fragestellungen des symbolischen Rechnens und Erstellung
von Werkzeugen, mit denen diese Daten bequem gewartet und erweitert
werden k"onnen.
\item Entwurf von Konzepten und Implementierung von Werkzeugen, unter
denen sich glaubw"urdige Benchmark-Rechnungen auf diesen Daten
ausf"uhren lassen.
\item Bereitstellung von Werkzeugen, mit denen man in den gesammelten
Daten navigieren, diese nach verschiedenen Kriterien selektieren und
in andere Formate transformieren kann.
\end{enumerate}

In der ersten Phase des Projekts konzentrierten wir uns darauf,
allgemeine Designprinzipien zu fixieren, die einerseits eine hohe
Flexibilit"at und Erweiterbarkeit garantieren, andererseits aber auch
gen"ugend einfach und praktisch in ihrer Handhabung sind.

Wir haben uns dabei bem"uht, stets die Umsetzung der Konzepte in
Werkzeuge und die Anwendbarkeit der Werkzeuge auf gr"o"sere
Datenmengen zu erproben, so dass zum gegenw"artigen Zeitpunkt neben
hinreichend bew"ahrten Konzepten auch eine ganze Reihe von
Werkzeugen und Datensammlungen zu zwei Benchmarkbereichen --
polynomialen Systemen und mechanisiertem Theorembeweisen in der
Geometrie -- zur Verf"ugung stehen. 

\subsection*{Grunds"atzliche "Uberlegungen}

Vom Ansatz her ist eine solche Datensammlung eine
Datenbank-Applikation, wobei uns auf Grund der spezifischen Anwendung
ein {\bf objekt-relationales Konzept} besonders geeignet erschien.
Mit einem solchen Ansatz kann man einerseits Querverbindungen zwischen
verschiedenen Daten, etwa zwischen Problembeschreibungen, Ergebnissen,
Hintergrundinformationen und Literaturquellen, gut erfassen, zum
anderen aber auch g"unstig modulare, objektorientierte Konzepte bei
der Erstellung der Werkzeuge f"ur die Manipulation der verschiedenen
Datenklassen verwirklichen.

Vor allem aus Gr"unden der Flexibilit"at haben wir uns, wenigstens im
Moment, entschieden, f"ur die Speicherung der prim"aren Quellen
unserer Daten keine klassische Datenbank zu nutzen, sondern diese
direkt in einem {\bf XML-artigen ASCII-Format} abzulegen. Wir gingen
bei dieser Entscheidung davon aus, dass relevante Daten der
Computeralgebra in Form von Strings vorliegen bzw.\ bequem als solche
gespeichert werden k"onnen.  Im vorliegenden Format sind sie auch
reinen ASCII-Editoren zug"anglich, aber zugleich in einer Weise
gespeichert, die Anforderungen k"unftiger XML-Werkzeuge nahe kommt.

Die Verarbeitung und Manipulation dieser Daten erfolgt mit {\bf
Perl-Werkzeugen}. Perl ist eine Programmiersprache aus dem
Open-Source-Bereich, die sich mit ihren leistungsstarken
String\-manipulations- und Skriptingf"ahigkeiten f"ur diese Aufgaben
als sehr geeignet erwiesen hat. Zur Strukturierung der Werkzeuge wird
intensiv von den Modularisierungsm"oglichkeiten Gebrauch gemacht, die
Perl 5 bereit stellt.

Sowohl aus Gr"unden der Flexibilit"at als auch mit Blick auf
potentielle Nutzer bleibt Perl aber im Hintergrund. Alle wichtigen
Standard-Applikationen k"onnen "uber eine gut dokumentierte, flexible
und intuitiv zu bedienende {\bf einheitliche Schnittstelle} im
"ublichen Kommandozeilen-Format ausgef"uhrt werden. 

Um die notwendige Flexibilit"at bei der Erweiterung und Modifikation
bereits bestehender Datenklassen ({\em Tabellen} in der Sprache der
Datenbanken) sowie der Erzeugung neuer solcher Tabellen, etwa f"ur
weitere Benchmark-Bereiche, zu erreichen, ist die entsprechende
Strukturinformation nicht in Perl festgeschrieben. Statt dessen lesen
die Werkzeuge zur Laufzeit diese Informationen aus {\bf Meta-Tabellen}
ein, deren Attribute im selben XML-ASCII-Format vorliegen wie die
Datenrecords selbst.  Sogar das {\bf Typkonzept}, das die Erstellung
solcher Attributbeschreibungen unterst"utzt, liegt in diesem Format
vor und kann damit leicht, unabh"angig von den Werkzeugen und ohne
Perlkenntnisse modifiziert und erweitert werden.

\subsection*{Der aktuelle Stand}

Die hier vorgestellten konzeptionellen "Uberlegungen standen so
nat"urlich noch nicht von Anfang an fest, sondern sind im Laufe der
Zeit und intensiver Arbeit an den bisher gesammelten Daten und bei der
Implementierung der nun verf"ugbaren Werkzeuge entstanden.  Obwohl die
Konzepte inzwischen eine gewisse innere Konsistenz aufweisen, sind
zuk"unftige gut begr"undete "Anderungen und Modifikationen, wie bei
jedem Software-Projekt, nicht auszuschlie"sen.  Wir gehen allerdings
davon aus, dass die "uber 1100 Datenrecords, die wir aus zwei gro"sen
Benchmark-Bereichen, dem Bereich der polynomialen Gleichungssysteme
und dem Bereich des mechanisierten Geometrie-Theorembeweisens,
gesammelt und mit unseren Werkzeugen erfolgreich gewartet und
manipuliert haben, eine Gew"ahr daf"ur bieten, dass diese "Anderungen
eher evolution"aren Charakter haben werden. 

Das \SD-Team hat neben diesen 1100 Datenrecords mehr als 40
Perl-Module mit "uber 15\,000 Zeilen Code geschrieben und "uber 20
Aktionen in der einheitlichen Schnittstelle zusammengef"uhrt. Der
folgende kurze alphabetische "Uberblick "uber die existierenden
Tabellen mag den Lesern ein Gef"uhl f"ur die Gesamtstruktur der bisher
gesammelten Daten geben:
\begin{description}
\item[Tabelle BIB] sammelt bibliographische Informationen im
BibTeX-Format, kurze Abstracts und Querverweise zu den Tabellen {\tt
GEO}, {\tt INTPS} und {\tt PROBLEMS}.

\item[Tabelle CAS] enth"alt allgemeine Beschreibungen einzelner
Computer-Algebra-Software.

\item[Tabelle CASCONFIG] enth"alt Konfigurationen einzelner CAS, die
f"ur verschiedene Bench\-mark-Rechnungen erforderlich sind und
Querverweise zu den Tabellen {\tt CAS} und {\tt COMP}.

\item[Tabelle COMP] enth"alt Informationen "uber die
auszuf"uhrenden Benchmarkrechnungen selbst.

\item[Tabelle COMPREPORT] enth"alt Reports von
Benchmark-Rechnungen. 

\item[Tabelle COMPRESULTS] enth"alt Resultate von
Benchmark-Rechnungen.

\item[Tabelle GEO:] Eine Sammlung von Problemen aus dem Bereich des
mechanisierten Geo\-metrie-Theorembeweisens mit Querverweisen zu den
Tabellen {\tt INTPS} und {\tt PROBLEMS}.

\item[Tabelle INTPS:] Eine Sammlung von Problemen aus dem Bereich der
polynomialen Gleichungssysteme mit Querverweisen zu den Tabellen {\tt
BIB} und {\tt PROBLEMS}.

\item[Tabelle MACHINE:] Zusammenstellung der Computer, auf denen
Benchmark-Rechnungen ausgef"uhrt wurden. Querverweise zur Tabelle {\tt
CASCONFIG}.

\item[Tabelle PERSON:] Informationen "uber die am \SD-Projekt
beteiligten Personen.

\item[Tabelle PROBLEMS:] Detailliertere Hintergrund-Informationen
und Kommentare zu einzelnen Problemen, etwa eine Beschreibung, ein
Verweis auf Originalarbeiten, relevanter CAS-Code und/oder einige
Schl"usselworte.
\end{description}

Wir haben mit ersten Benchmark-Rechnungen zu Gr"obnerbasen begonnen,
die verschiedene Koeffizientenbereiche und Termordnungen verwenden.
Diese Benchmark-Rechnungen laufen mit 10 Versionen verschiedener CAS
auf den "uber 500 {\tt INTPS}-Records. Weitere Benchmark-Rechungen
sind in Vorbereitung, wobei wir auf die exzellenten Bedingungen am UMS
MEDICIS (\url{www.medicis.polytechnique.fr/medicis}) zur"uckgreifen
k"onnen.  Auch unsere von der Fachgruppe CA gesponsorte Web-Site ist
dort physisch beheimatet.

\subsection*{Wie weiter?}

Aus unserer Sicht sind die Konzepte und Werkzeuge inzwischen so weit
gereift, dass sie einem gr"o"seren Personenkreis zur Begutachtung und
Weiterentwicklung zur Verf"ugung gestellt werden k"onnen. Eine
Weiterentwicklung wird vor allem dann notwendig werden, wenn sich aus
Anwendungen in anderen als den bisher untersuchten Benchmark-Bereichen
Stellen ergeben sollten, an denen die Konzepte noch nicht allgemein
genug sind.  Wir suchen deshalb den Kontakt zu Gruppen, die solche
Anwendungen studieren, "uber relevante Datenmengen verf"ugen und
bereit sind, diese unter den Bedingungen der Gnu Public License mit
anderen zu teilen.

Obwohl wir dies auf ein Minimum zur"uckgeschraubt haben, wird man beim
Erschlie"sen neuer Benchmark-Bereiche, besonders f"ur semantische
Aspekte, nicht ohne ein gewisses Ma"s an Perl-Programmierung
auskommen.  Das \SD-Team steht mit Rat und im Rahmen seiner
M"oglichkeiten auch mit tatkr"aftiger Unterst"utzung bereit, diesen
Programmieraufwand zu bew"altigen.  Allerdings wird das Projekt auf
l"angere Sicht nur erfolgreich sein k"onnen, wenn es auch personelle
Verst"arkung erf"ahrt, die wir gern unb"urokratisch und zu
gleichberechtigten Bedingungen in unser Team integrieren.

\end{document}

