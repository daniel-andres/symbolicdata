\documentclass{svmult}
\usepackage{url}

\newcommand{\SD}{\textsc{Symbolic\-Data}}

\title*{The {\SD} Project} 

\author{Hans-Gert Gr\"abe \and Andreas Nareike\and Simon Johanning}

\institute{Hans-Gert Gr\"abe \and Andreas Nareike\and Simon Johanning \at
  Universit\"at Leipzig, Germany,
  \email{(graebe|nareike|johanning)@informatik.uni-leipzig.de}}
\begin{document}
\maketitle

\abstract{We report about a major reengineering of the {\SD} project \cite{SD}
  and its redesign according Linked Data principles and RDF technologies that
  proved to be powerful within modern semantic web approaches. During that
  redesign the focus of the project changed from a mere data store towards the
  vision of a Computer Algebra Social Network that aims at technical support
  of the intercommunity communication between different CA subcommunities. The
  redesign was supported by Saxonian E-Science Initiative grants for
  A.~Nareike and S.~Johanning. A first milestone was the release of version~3
  of {\SD} in September 2013.}

\section{Introduction}

A powerful digital research \emph{infrastructure} becomes increasingly
important in today's networked and interlinked world.  This includes digital
support for dissemination of new papers, the refereeing process, conference
submissions, and scientific communication within communities.  Services such
as MathSciNet, arXiv.org, EasyChair.org, or bibsonomy.org have been
established and their usefulness is acknowledged by the larger scientific
community.

Practical efforts to build up digital support for research infrastructures can
be traced back at least until the beginning of the 1990th in different areas
of science. E.g., in Computer Algebra (CA) the EU funded projects PoSSo
\cite{PoSSo} and FRISCO \cite{FRISCO} aimed at developing such a research
infrastructure in the area of Polynomial Systems Solving.

Although being a challenging engineering task such efforts are rarely
acknowledged by the reputational processes of science, and hence are left to
the casual engagement of volunteers.  Different to the infrastructural
processes in the ``large science'' for small academic communities as, e.g.,
the CA community, it is hard to find funding for special infrastructural
projects and they have to organize the development of their
\emph{intracommunity} communication infrastructure in a different, more
distributed way.

Open Source culture offers plenty of experience of how to substitute centrally
organized projects by decentralized networked structures and indeed the new
focus of {\SD} version 3 \cite{SD} is that of an intercommunity project,
providing not only reliable access to data for testing and benchmarking
purposes but also technical support for interlinking between different CA
subcommunities.

In this paper we present our \emph{e-science environment engineering} approach
to redesign {\SD} in the indicated way with scarce resources (less than a
man-year).  In such a setting we had to focus on communication with other
existing activities and on real engineering aspects along a roadmap
predetermined by the international development of e-science standards, in
particular by the Dublin Core Metadata Initiative \cite{dc}, the Virtual
International Authority File \cite{VIAF}, and the corresponding German GND
activities \cite{GND}.  Our main goal was to produce and operate a valuable
piece of software. In the scope of this conference we can only give a short
report about our special e-science engineering task with focus on
interoperability with the large e-science world.

We assume the reader to be familiar with the basic terminology in that area,
in particular with the \emph{Resource Description and Access} (RDA) concepts
and standards \cite{RDA}.

\section{Background}

{\SD} has been part of CA infrastructural efforts for almost 15 years.  It
grew out of a Special Session on Benchmarking at the 1998 ISSAC conference,
where the participants were faced with a typical situation: within the PoSSo
and FRISCO projects volunteers compiled a large database of Polynomial Systems
with the goal to make it publicly available for testing and benchmarking of
algorithms.  At the end of the project people switched to other tasks and it
became more and more problematic to access the data.  Even worse, badly cloned
copies of the data were disseminated and after a while it was even hard to
decide, e.g., what exactly `Katsura-5' means -- is it about the example from
the well-known series with 5 variables $y_1,\dots,y_5$ or with 6 variables
$x_0,\dots,x_5$?

The {\SD} Project started in 1999 on that basis to build a reliable and
sustainably available reference of Polynomial Systems data, to extend and
update it, to collect metainformation about the records, and also to develop
tools to manage the data and to set up and run testing and benchmarking
computations on the data.  A first implementation with data from Polynomial
Systems Solving and Geometry Theorem Proving was realized by Olaf Bachmann and
Hans-Gert Gr\"abe in 1999 and 2000.  Data from other fields, were added later
on by the CoCoA and Singular teams, V.~Levandovskyy, and R.~Hemmecke, and the
Web site \url{symbolicdata.org} sponsored by the `Fachgruppe Computeralgebra'
went online in 2005.  

Later on the project joined forces with the Agile Knowledge Engineering and
Semantic Web (AKSW) Group at Leipzig University \cite{AKSW} to strongly
refactor the data along standard Semantic Web concepts based on the Resource
Description Framework (RDF).  These efforts were endorsed in 2012/13 by a
Saxonian E-Science Initiative grant \cite{E-Science-Sachsen} for A.~Nareike
and in 2014 for S.~Johanning.  A first milestone of this reengineering project
was the release of {\SD} version~3 in September 2013.

\section{The {\SD} Infrastructure}

Our resources (examples for testing, profiling and benchmarking software and
algorithms from different areas of symbolic computation) are publicly
available in XML markup, meta data in RDF notation both from a public git repo
at \texttt{http://github.org/symbolicdata}, and from an OntoWiki
\cite{OntoWiki} based RDF data store at \url{http://symbolicdata.org/Data}.
Moreover, we offer a SPARQL endpoint at
\url{http://symbolicdata.org:8890/sparql} to explore the data by standard
Linked Data methods.

The website operates on a standardized installation using an Apache web server
to deliver the data, the Virtuoso RDF data store \cite{Virtuoso} as data
backend, a SPARQL endpoint and (optionally) OntoWiki to explore, display and
edit the data.  This installation can easily be rolled out at a local site to
support local testing, profiling and benchmarking.  For details we refer to
the {\SD} wiki \cite{SD}.

The distribution offers also tools for integration with a local compute
environment as, e.g., provided by Sagemath \cite{Sagemath} -- the Python based
\emph{SDEval package} \cite{sdeval} by Albert Heinle offers a JUnit like
framework to set up, run, log, monitor, and interrupt testing and benchmarking
computations, and the \emph{SDSage package} \cite{sdsage} by Andreas Nareike
provides a showcase for {\SD} integration with the Sagemath compute
environment.

We follow a development process along the Integration-Manager-Workflow Model.
This makes it easy to join forces with the {\SD} team: Fork the repo to your
github account, start development and send a pull request to the Integration
Manager if you think you produced something worth to be integrated into the
upstream master branch.  Even if your contribution is not pulled to the
upstream, people can use it, since they can pull it from your to their github
repo. This allows even for agile common small feature development -- a widely
practised way to advance projects hosted at \texttt{github.com}. You are
encouraged early to start a discussion about your plans and regularly report
your progress on the {\SD} mailing list.

\section{{\SD} at Work}

Preparing {\SD} version 3 we decided to strengthen the {\SD} part that is
\emph{not} involved with Polynomial Systems Solving.  These efforts led to a
more consequent distinction between data (owned and maintained by different CA
subcommunities) and meta data. Such a distinction is well supported by RDF
design principles -- the Resource Description Framework is about
\emph{description} of \emph{resources}, represented by (globally unique)
\emph{resource identifiers} (URIs). We follow the Linked Data best practise to
provide URIs accessible by the HTTP internet protocol and to deliver a
valuable part of structured information about that resource upon HTTP request.

\subsection{Resources}

Currently the {\SD} data collection contains resources from the areas of
Polynomial Systems Solving (390 records, 633 configurations), Free Algebras
(83 records), G-Algebras (8 records), Geometry Proof Schemes (297 records) and
Test Sets from Integer Programming (28 records). These resources are stored
per file in a flat XSchema based XML syntax using well established
intracommunity syntaxes for the internal data.

Such a concept is not restricted to centrally managed resources, but can
easily be extended to other data stores on the web that are operated by
different CA subcommunities and offer a minimum of Linked Data facilities.
There are draft versions of resource descriptions about Fano Polytopes (8630
records) and Birkhoff Polytopes (5399 records) hosted by Andreas Paffenholz
and about Transitive Groups (3605 records) from the Database for Number Fields
of J\"urgen Kl\"uners and Gunter Malle that point to external resources. 

\subsection{Metadata and Resource Descriptions}

It was one of the great visions of the {\SD} Project to collect not only
benchmark and testing data but also valuable background information about the
records in the database as, e.g., information about papers, people, history,
systems etc.\ concerned with the examples in our collection.  Within the
redesign we developed a general concept of the \texttt{sd:Annotation} RDF
class to store background information in a unified way.  We use that concept
in particular to relate bibliographical entries of type \texttt{sd:Reference}
to different data records.

The management of bibliographical references was completely redesigned with
{\SD} version 3 exploiting RDF and the established Dublin Core ontology
\cite{dcterms} to represent bibliographical information in a way that is
queryable by standard means and tools. On the other hand, we strongly reduced
the part of information about bibliographical references kept inside {\SD} to
a toe-hold since there are comprehensive bibliographical stores available on
the web that provide all required information and there is no need to spend
efforts to collect such information twice, even if the foreign repositories
are not (yet) querable by RDF techniques but provide only access to permanent
URLs and to local search engines to extract the required information. 

\subsection{Towards a Computer Algebra Social Network}

From the five stars to be assigned to a Linked Data project according to Tim
Berners-Lee's classification \cite{5stars} {\SD} earned four stars so far (for
offering data in interoperable RDF format on the web and providing a SPARQL
querable RDF triple store).  For the fifth star one has to build up stable
semantic relations to foreign knowledge bases and thus become part of the
Linked Open Data Cloud \cite{lod}.

Much of such interrelation, e.g., a list of interoperability references for
people, software and bibliographical data with the Zentralblatt, is on the
way.  Moreover, we joined forces with the efforts of the board of the German
Fachgruppe to store and provide information about people and groups working on
CA topics at their new Wordpress driven web site \cite{cafg}.  We developed a
first prototype \cite{sdcasn} to store this information in RDF format, to
extract it by means of SPARQL queries and to view it on the web site using the
Wordpress shortcode mechanism via a special Wordpress plugin.  We apply this
technique to maintain information about
\begin{itemize}
\item German working groups in CA,
\item projects within the SPP 1489 priority program,
\item graduate theses in CA and
\item upcoming conferences
\end{itemize}
at this site \cite{cafg}.  

The vision of a Computer Algebra Social Network goes far beyond that: Set up
and run within the CA community a semantic aware Facebook like Social Network
that provides RDF based tools in such a way that interested people can easily
contribute about all topics around Computer Algebra.  This sounds quite
visionary, but it is in no way utopic. We started to operate a very first
prototypical node \url{http://symbolicdata.org/xodx/} \cite{xodx} that
realizes the challenging concept of a \emph{Distributed Semantic Social
  Network} \cite{DSSN}.

\raggedright
\begin{thebibliography}{1}

\bibitem{AKSW} Agile Knowledge Engineering and Semantic Web (AKSW) Working
  Group at Leipzig University, \url{http://aksw.org/About.html}

\bibitem{xodx} Arndt, N.: Xodx: A basic DSSN node implementation,
  \url{http://aksw.org/Projects/Xodx.html}

\bibitem{5stars} Berners-Lee, T.: 5 stars for Open Data.
  \url{http://5stardata.info/}

\bibitem{dc} The Dublin Core Metadata Initiative, \url{http://dublincore.org/}

\bibitem{dcterms} DCMI Metadata Terms, 
  \url{http://dublincore.org/documents/dcmi-terms/} 

\bibitem{E-Science-Sachsen} Das eScience-Forschungsnetzwerk Sachsen, 
  \newblock \url{http://www.escience-sachsen.de} 

\bibitem{cafg} Website of the German Fachgruppe Computeralgebra,   
  \url{http://www.fachgruppe-computeralgebra.de/} 

\bibitem{FRISCO} FRISCO -- A Framework for Integrated Symbolic/Numeric
  Computation, \url{http://www.nag.co.uk/projects/FRISCO.html} 

\bibitem{GND} Gemeinsame Normdatei (GND), 
  \url{http://www.dnb.de/DE/Standardisierung/GND/gnd_node.html}

\bibitem{sdeval} Heinle, A.: The SDEval framework.  
  \url{http://symbolicdata.org/wiki/SDEval}

\bibitem{lod} Linked Data, \url{http://en.wikipedia.org/wiki/Linked_data}

\bibitem{sdsage} Nareike, A.: The SDSage package, 
  \url{http://symbolicdata.org/wiki/PolynomialSystems.Sage}

\bibitem{OntoWiki} OntoWiki -- A tool providing support for agile, distributed
  knowledge engineering scenarios, 
  \url{http://aksw.org/Projects/OntoWiki.html} 

\bibitem{PoSSo} The PoSSo Project, \url{http://posso.dm.unipi.it/}

\bibitem{RDA} The Joint Steering Committee for Development of RDA, 
  \url{http://www.rda-jsc.org/rda.html}

\bibitem{Sagemath} Sage -- a free open-source mathematics software system, 
  \url{http://www.sagemath.org} 

\bibitem{SD} The SymbolicData Project Wiki, \url{http://symbolicdata.org}

\bibitem{sdcasn} The SymbolicData CASN RDF triple store, 
  \url{http://symbolicdata.org/casn/} 

\bibitem{DSSN} Tramp, S. Frischmuth, P., Ermilov, T., Shekarpour, S., Auer,
  S.: An Architecture of a Distributed Semantic Social Network.  Semantic Web
  Journal (2012),
  \url{http://www.semantic-web-journal.net/sites/default/files/swj201_4.pdf} 

\bibitem{VIAF} The Virtual International Authority File,
  \url{http://www.oclc.org/viaf.en.html}

\bibitem{Virtuoso} Virtuoso Open-Source Edition,
  \url{http://virtuoso.openlinksw.com/}

\end{thebibliography}
\end{document}
