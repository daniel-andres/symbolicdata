\documentclass{article}
\usepackage{a4wide,german,url}
\usepackage[utf8]{inputenc}

\newcommand{\SD}{{\sc Symbolic\-Data}}
\parindent0pt
\parskip4pt

\begin{document}

\section*{\centering Neues vom \SD-Projekt}

\begin{center} 
Hans-Gert Gr"abe (Leipzig)\footnote{published in ``Computeralgebra-Rundbrief''
  51, October 2012}
\end{center}

\subsection*{Zur Geschichte des Projekts}

Die Idee zum \SD-Projekt entstand 1998 am Rande der ISSAC in Rostock, als
zunehmend sichtbar wurde, dass die Datensammlungen, welche im Rahmen der mit
europäischen Mitteln geförderten Projekte
PoSSo\footnote{\url{http://posso.dm.unipi.it}} und
FRISCO\footnote{\url{http://www.nag.co.uk/projects/frisco.html}} entstanden
waren, ohne aktives Zutun der Community nicht dauerhaft digital verfügbar sein
würden.

In den folgenden Jahren wurde dieses Projekt vor allem von Olaf Bachmann und
dem Autor dieses Beitrags mit weiterer Unterstützung aus der Gruppe der
Singular-Entwickler vorangetrieben. Im CAR 26 (M"arz 2000) und CAR 28 (M"arz
2001) wurde das \SD-Projekt im damaligen Zuschnitt ausf"uhrlich vorgestellt.
Der Schwerpunkt der Datensammlung lag im Bereich polynomialer
Gleichungssysteme mit ganzzahligen Koeffizienten, die zu jener Zeit
insbesondere im Zusammenhang mit verschiedenen Varianten des
Gröbner-Algorithmus intensiv studiert wurden, wobei vor allem die
verschiedenen Benchmark-Beispiele aus der Literatur und den genannten
EU-Projekten systematisiert und digital aufbereitet sowie entsprechende
Werkzeuge zur Verwaltung und zum Aufsetzen von Benchmarks entwickelt wurden.
 
Nach dem Weggang von O.~Bachmann aus Kaiserslautern zum Ende 2000 wurde mit
dem geometrischen Theorembeweisen ein zweites größeres Sammelgebiet
erschlossen, das über die Koordinatenmethode eng mit polynomialen
Gleichungssystemen verbunden ist. Einfache Beispiele lassen sich auf
geometrische Sätze \emph{vom konstruktiven Typ} zurückführen, deren Beweis auf
eine {\glqq}einfache{\grqq} Normalformberechnung rationaler Ausdrücke
zurückgeführt werden kann. Obwohl solche Normalformberechnungen im Prinzip
verstanden sind, lassen sie sich aber praktisch oft nicht ausführen, während
alternative Koordinatisierungen \emph{vom Gleichungstyp} und geschickte
Ansätze des Lösens der entsprechenden polynomialen Gleichungssysteme zum Ziel
führten. M.~Witte, damals Student in Leipzig, untersuchte und digitalisierte
insbesondere eine größere Anzahl von Beispielen aus der Beispielsammlung von
S.-C.~Chou\footnote{S.-C. Chou. Mechanical Geometry Theorem Proving. Reidel,
  Dortrecht, 1988.}. 

Weiteres Beispielmaterial wurde von der CoCoA-Gruppe (F.~Cioffi), der
Singular-Gruppe (M.~Brickenstein, M.~Dengel, S.~Steidel, M.~Wenk),
J.-C.~Faugère (weitere Polynomsysteme), V.~Levandovskyy (nichtkommutative
Polynomsysteme, G-Algebren) sowie von Raymond Hemmecke (Testmengen aus der
ganzzahligen Programmierung) beigesteuert.  Die Projektdaten wurden gemeinsam
zunächst über ein CVS-Repository verwaltet.  Die CA-Fachgruppe finanziert für
das Projekt die Domäne \texttt{symbolicdata.org}, die seit 2005 von der
GI-Geschäftsstelle gehostet wird.

Leider standen ab 2003 keine personellen Ressourcen zur systematischen
weiteren Arbeit am \SD-Projekt zur Verfügung, so dass das Datenmaterial
seither nur auf einer {\glqq}as is{\grqq} Basis verwendet werden konnte und
auch nur gelegentliche Erweiterung erfahren hat. Insbesondere stellte sich
heraus, dass die ambitionierten Vorstellungen, eine \emph{einheitliche
  Werkbank für Benchmarking im symbolischen Rechnen} mit entsprechenden
Werkzeugen und Konzepten zu entwickeln, mit den verfügbaren Ressourcen nicht
zu stemmen war und -- mit Blick auf die Vielfalt der Rechnerarchitekturen,
Problemstellungen und verwendeten Programmiersprachen -- wohl auch am realen
Bedarf vorbei geht. Statt dessen konzentriert sich dieser Teil des Projekts
nunmehr darauf, den Code von Best Practise Beispielen zu sammeln und zur
Nachnutzung zur Verfügung zu stellen.  

Ein weiterer Versuch, die Arbeiten am \SD-Projekt mit vereinten Anstrengungen
zu intensivieren und mit angrenzenden Bemühungen in dieser Richtung, etwa der
Datenbank \emph{Gröbner Bases Implementations. Functionality Check and
  Comparison}\footnote{\url{http://www.risc.jku.at/Groebner-Bases-Implementations}}
oder dem \emph{Gröbner Bases Bibliography
  Project}\footnote{\url{http://www.risc.jku.at/Groebner-Bases-Bibliography}}
zu koordinieren, wurde im März 2006 während des \emph{Special Semester on
  Groebner Bases} am RISC Linz gestartet.  Jedoch fand auch dieses Projekt im
Nachgang keine nachhaltige Resonanz in der Community. 

\subsection*{Entwicklungen nach 2003}

In den letzten 10 Jahren nahm das (semantische) Web der Daten eine besonders
stürmische Entwicklung, was die konzeptionellen Entwicklungen zur Datenhaltung
aus den Anfangszeiten des \SD-Projekts entwertet hat, über die seinerzeit in
den entsprechenden Aufsätzen im CAR berichtet wurde.  Mit dem \emph{Semantic
  Stack} haben sich inzwischen Standards durchgesetzt und stehen Werkzeuge zur
Verfügung, mit denen sich Verwaltungsaufgaben auf verschiedenen Ebenen nach
einheitlichen Prinzipien organisieren lassen. Dies sind 
\begin{itemize}
\item XML und XSchema als Standards zur Darstellung strukturierter Daten auf
  Ressourcen-Ebene sowie 
\item RDF und OWL als Standards zur Beschreibung semantischer Aspekte der
  Daten.
\end{itemize}
Während ersteres prinzipiell geeignet ist, dezentral betreute Datenbestände in
einem \SD-Netz zusammenzuführen und durch geeignete URI's zu adressieren (die
Frage der nachhaltigen Verfügbarkeit derartiger dezentral betreuter
Datenbestände sei hier ausgeklammert), ist die Aggregierung der Beschreibungen
semantischer Aspekte unbedingt erforderlich, um sinnvolle Suchprozesse nach
Informationen auf diesen vernetzten Beschreibungen zu ermöglichen.  Während
wir in der Anfangszeit davon ausgingen, dass dazu wesentliche Teile der
Informationen an einem Ort zusammengetragen sein muss -- etwa als Datenbestand
in unserem CVS-Repository -- zeigen modernere Entwicklungen der Linked Open
Data Cloud, in welchem Umfang auch diese Informationen dezentral verwaltet
werden können, wenn entsprechende Protokolle des Datenaustauschs vereinbart
werden. Genauer geht es nicht um die Protokolle selbst, sondern um die
gemeinsam zu vereinbarenden Ontologien, mit denen RDF-Tripel der Wissensbasis
auf einheitliche Weise {\glqq}verstanden{\grqq} werden.

Die wenigen verfügbaren Ressourcen der letzten Jahre wurden darauf verwendet,
einen Migrationsprozess der Datenbasis hin zu diesen Standards in die Wege zu
leiten, also 
\begin{itemize}
\item die Benchmark-Daten aus den verschiedenen Bereichen auf je
  standardisierte Weise in XSchema-beschriebene XML-Ressourcen zu
  transformieren, 
\item die im ursprünglichen \SD-Konzept eng mit den Ressourcen gekoppelten
  Beschreibungsdaten zu separieren und über RDF-Standards (insbesondere einen
  SPARQL-Endpunkt) für Suchanfragen verfügbar zu machen sowie 
\item eine \SD-Ontologie für diese Beschreibungsdaten zu entwickeln.
\end{itemize}
Die Erfahrungen aus vergleichbaren Projekten an unserem Lehrstuhl legten nahe,
dass dieser Migrationsprozess am besten als agiler Prozess anzulegen ist, also
das {\glqq}a little bit thinking, a little bit coding{\grqq} aus der
Anfangszeit des \SD-Projekts auch hierbei anwendbar ist, um mit kleinen
Schritten die sehr begrenzten Ressourcen einzusetzen.  Insbesondere lehren
diese Beispiele, dass es -- im deutlichen Gegensatz zu den Erfahrungen der
ER-Modellierung -- nicht sinnvoll ist, bereits am Anfang zu viel Mühe
auf die genaue Fassung der \SD-Ontologie

Im Repository sind größere Teile der Daten (insbesondere ein wesentlicher Teil
der Polynomsysteme) inzwischen in XSchema-beschriebene XML-Ressourcen
transformiert


http://hgg.ontowiki.net/



Seit Anfang 2011 wird Mercurial zur Projektverwaltung\footnote{Siehe
  \url{https://bitbucket.org/hgg/symbolicdata}} eingesetzt.

\subsection*{Das Benchmarking-Projekt im Rahmen der Sächsischen
  E-Science-Initiative} 

Projekt, symbolicdata Mailingliste

Aufbau eines \SD-Wiki 

Dort Fixierung der \SD-Ontologie als Kommunikationsprozess 

Weiter untersuchen, welches Potenzial die SemanticMediawiki-Erweiterung und
SemanticForms für das \SD-Projekt bieten.

Einbeziehung weiterer Datenbestände, wenigstens auf der Beschreibungsebene. 



\end{document}



Von mir wurde die bereits im Konzept angelegte Trennung von Daten- und
Programmteil (vgl. CAR 26) nun auch auf technischer Ebene realisiert,
wodurch es m"oglich ist, dieselben Werkzeuge f"ur verschiedene Projekte
zur Verwaltung symbolischer Informationen einzusetzen. Die jeweils
verwendete Datenbasis kann dazu "uber den Wert einer Option eingestellt
werden.  

Zur Verwaltung symbolischer Daten kann zwar auch eine herk"ommliche
Datenbank eingesetzt werden, jedoch erlaubt unser objekt-relationaler
Ansatz zusammen mit der Flexibilit"at von Perl einen deutlich
einfacheren Umgang mit ASCII-Quellen (bis hin zur M"oglichkeit, diese
'in situ' zu korrigieren), so dass sich die \SD-Werkzeuge auch
``artfremd'' (ich verwende sie zum Aufbau einer Literatur-Datenbank
sowie beim Management von Olympiade-Aufgaben) einsetzen lassen.
\medskip

Die Resonanz auf unsere Arbeit innerhalb der CA-Gemeinde ist jedoch
immer noch gering.  Ich weise deshalb noch einmal auf die Anliegen
hin, die wir mit unserem Projekt verfolgen: 
\begin{enumerate}
\item[1.] Wir wollen die Bem"uhungen verschiedener Gruppen zur
Erstellung von Perl-Werkzeugen zum Management digitaler symbolischer
Daten aus verschiedenen Bereichen der Computeralgebra vereinigen.
\end{enumerate}

Wir gehen davon aus, dass auch andere Gruppen sich eigene Werkzeuge f"ur
Vergleichs- oder Testrechnungen erstellt haben oder erstellen und hier bereits
mehrfach das Fahrrad neu erfunden wurde und auch weiter erfunden werden wird,
so dass es an der Zeit ist, das ganze know how einmal zu sichten und zu
b"undeln.

Unsere Werkzeuge (provided 'as is') k"onnen von der \SD-Webseite
\url{http://www.symbolicdata.org} herunter geladen und leicht f"ur spezielle
Zwecke in einem lokalen Projekt angepasst und modifiziert werden.  Eine
sinnvolle Weiterentwicklung ist erst auf der Basis eines so gewonnenen
Erfahrungsschatzes m"oglich.

\begin{enumerate}
\item[2.] \SD\ stellt ein zentrales Repositorium zur Verf"ugung, in dem
digitale Benchmark-Daten aus verschiedenen Bereichen der
Computeralgebra gesammelt werden (k"onnen).
\end{enumerate}

Ein solcher {\em upload} ist derzeit nur "uber eine direkte Beteiligung am
Projekt m"oglich, in dessen Rahmen Zugang zu unserem CVS-Repository am UMS
MEDICIS in Paris (\url{http://www.medicis.polytechnique.fr/medicis}) besteht.
\medskip

Eine stabile Version 0.4 der zur Zeit verf"ugbaren Werkzeuge und Daten
ist ab 1.~M"arz~2001 auf unserem Server verf"ugbar.  Teilbereiche des
Projekts warten auf die weitere Ausgestaltung (z.B. ein
Frontend/Backend-System, um die Ladezeiten zu verringern; ein auf
"Ahnlichkeit basierendes System von Vergleichen von Records; dynamische
Webseiten-Generierung aus der Datenbasis; ein vern"unftiger
Select-Mechanismus), wozu weitere Mitstreiter willkommen sind.

\end{document}
