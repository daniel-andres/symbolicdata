%% Note that the length of the abstract should not exceed one page
\documentclass{dmv-ptm}

%% This is a place for LaTeX packages you need
\usepackage{amsmath}
\usepackage{amsthm}

%% This is a place for your own commands



%%
\title{swMATH � a new information service on mathematical software}
\author{Wolfram Sperber}
\institution{zbMATH - FIZ Karlsruhe}
\country{Germany}
\email{wolfram@zentralblatt-math.org}

%% the following command is optional: if you don't need leave %

\coauthor{The talk is based on joint work with Gert-Martin Greuel }

%% Enter the session title - you can find it on the web page
%% http://ptm-dmv.wmi.amu.edu.pl/Symposia-accepted.html

\session{Information and Communication in Mathematics}

\begin{document}

\maketitle

\begin{abstract} %% put your abstract below

Mathematical software is an emerging field of mathematical knowledge and a important bridge between mathematical research and mathematical applications. But searching mathematical software information is not easy: the information about software is often sparse and  heterogeneous, standards for the description of mathematical software and specialized information services are missing. 

The swMATH service is a new and efficient approach for developing an information service about mathematical software basing on the systematic reference analysis of mathematical publications. The swMATH service covers currently nearly 7,000 mathematical software packages with detailed information, in particular with links to more than 69,000 research articles in Zentralblatt MATH (zbMATH), referring to a package in swMATH.

The talk discusses the concept and the state of the art of swMATH.
       
%% References are optional, follow the format
% \begin{thebibliography}{22}
% \bibitem{ref-01} T.Tao, \emph{Higher order Fourier analysis}, 
% � Studies in Mathematics 142, American Mathematical Society,
% � Providence, RI 2012.
% \bibitem{ref-02} B. Green, T. Tao, \emph{The quantitative behaviour 
% � � polynomial orbits on nilmanifolds}, Ann. of Math. �175, 2012,
% � 465--540.
% \end{thebibliography}

\end{abstract}
\end{document}

