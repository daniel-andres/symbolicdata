\documentclass[11pt,a4paper]{article}
\usepackage{a4wide,url}

\parindent0pt
\parskip4pt

\begin{document}
\begin{center}
  \Large About the Migration to the New SymbolicData Format

  \normalsize Version of August 24, 2012
\end{center}

\section{Background}

The SymbolicData project grew out of the special session on Benchmarking at
the 1998 ISSAC conference in Rostock which was organized by H.~Kredel. In the
next three years, the project has steadily developed from ideas to
implementations and data collections and back.

In 1999, the project joint forces with the symbolic computation groups of the
University of Paris VI (J.\,C. Faugere, D. Lazard), of Ecole Polytechnique
(J. Marchand, M. Giusti), and of the University of Saarbrücken (M. Dengel,
W. Decker). Furthermore, the project was incooperated into the benchmarking
activities of the \emph{Fachgruppe Computeralgebra} of the Deutsche
Mathematiker-Vereinigung (Chair at those times: G.-M. Greuel).

The main design decisions and implementations of the first prototype were
realized by O. Bachmann and H.-G. Gräbe during two visits in Leipzig and
Kaiserslautern in 1999 and 2000. The data was stored in a flat XML-like syntax
and managed with elaborated Perl tools. Meta information about data was stored
in the same format in a special META directory and allowed for a unique and
flexible handling of all the data.

We collected data from \emph{Polynomial System Solving} and \emph{Geometry
  Theorem Proving}, set up a CVS repository, and started test computations,
both at UMS Medicis, with the main focus on Polynomial System Solving. The
prototype was presented at the Meeting of the Fachgruppe Computeralgebra,
Kaiserslautern, February 2000.

O. Bachmann left the project for a new job at the end of 2000. Now the main
focus moved to Geometry Theorem Proving. M. Witte (Leipzig) digitized a great
part of the 512 geometry theorems considered in [Chou 88] and prepared them
for benchmark computations with the GeoProver package of H.-G. Gräbe.

The project was presented in talks at RWCA-02, ADG-02 and also in the
CA-Rundbrief published by the Fachgruppe Computeralgebra. But there was not
enough interest to really push the project during 2002--2005. 

The main efforts in those times were spent on a further development of
concepts. First, the strong development and standardization of XML as markup
language suggested to move the format of data storage to a truly XML-based
design and have the META information encoded as XSchema. Second, with the
raise of the Semantic Web new concepts of ontology design came into the focus
of SymbolicData. They allow for a much more flexible handling of relational
information compared to the the former concept based on META information or
XSchema (and are also superior to database concepts).

Since 2005 the Web site \url{www.SymbolicData.org} is hosted by the
GI\footnote{Gesellschaft f\"ur Informatik, the ``German ACM''.}  office in
Bonn on behalf of the Fachgruppe Computeralgebra. During the Special Semester
on Groebner Bases in March 2006 we tried -- with little success -- a relaunch
of the project and to unite forces with the GB-Bibliography project
(B. Buchberger, A. Zapletal) and the GB-Facilities project (V. Levandovskyi).

Since those times H.-G. Gräbe took efforts to refactor the data along a new
concept of XMLResources and OWLResources as described on the Design
description page.


\section{Goal of the Migration}




\end{document}
